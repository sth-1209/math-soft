\documentclass{ctexart}

\usepackage{graphicx}
\usepackage{amsmath}

\title{作业一: 我的工作环境叙述}
\author{申屠慧 \\ 能源与环境系统工程(智慧能源班) 3210103417}
\date{\today}

\begin{document}

\maketitle


\section{我的计算机}
\noindent
XPS 13 9305\\
设备名称	DESKTOP-09DGDU6\\
处理器	11th Gen Intel(R) Core(TM) i7-1165G7 @ 2.80GHz   2.80 GHz\\
机带 RAM	16.0 GB (15.7 GB 可用)\\
设备 ID	049B28D1-C6E0-49B3-903B-1C60B79DBF26\\
产品 ID	00342-36173-59581-AAOEM\\
系统类型	64 位操作系统, 基于 x64 的处理器\\
笔和触控	笔支持\\
\\
硬盘大小 512GB\\
显卡型号 Intel(R)Iris(R) Xe Graphics\@

\section{我的Linux实现方式}
\noindent
虚拟机\\
Memery 3.8GiB\\
Disk Capacity 48.3GB\\  

\section{Linux版本与安装额外软件}
\noindent
Linux version 5.15.0-76-generic (buildd@lcy02-amd64-019)\\
 (gcc (Ubuntu 9.4.0-1ubuntu1~20.04.1) 9.4.0, 
 GNU ld (GNU Binutils for Ubuntu) 2.34) \#83~20.04.1-Ubuntu SMP Wed Jun 21 20:23:31 UTC 2023\\
 \\
安装软件:\\
texlive-full\\
Visual Studio Code\\
doxygen\\
g++\\
gfortran\\
make\\
git\\
libdealii\\
paraview\\
gnuplot\\
okular\\
googlepinyin\\

\section{编辑器和gcc编译器的版本}
\noindent
Visual Studio Code\\
1.79.2\\
695af097c7bd098fbf017ce3ac85e09bbc5dda06\\
x64\\
\\
gcc (Ubuntu 9.4.0-1ubuntu1~20.04.1) 9.4.0\\
Copyright (C) 2019 Free Software Foundation, Inc.\\
This is free software; see the source for copying conditions.  There is NO
warranty; not even for MERCHANTABILITY or FITNESS FOR A PARTICULAR PURPOSE.\\
\section{使用Linux环境工作的可能性和可能场景}
作为一名能源专业的工科生,我很可能在未来的学习与工作中应用linux环境,包括论文排版,科学计算,软件开发,机器学习等。\par
我希望在课程中可以入门linux系统,学习好latex语言,为接下来的科研训练奠定基础,提高效率。\\

\end{document}
